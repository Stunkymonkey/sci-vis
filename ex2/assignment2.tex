\documentclass[a4paper]{article}

\usepackage[ngerman]{babel} 
\usepackage[T1]{fontenc}    
\usepackage[utf8]{inputenc} 
\usepackage{textcomp}      
\date{} 					
\author{}                   
\usepackage{geometry}		
\geometry{ left=2cm, right=2cm, top=2cm, bottom=4cm, bindingoffset=5mm}

\usepackage{graphicx}
\usepackage{xcolor}
\usepackage{hyperref} 
\usepackage{fancyhdr}												
\pagestyle{fancy}
\fancyhf{}
\fancyhead[R]{2973140 - Felix Bühler  \\ 2893121 - Jan Leusmann \\  3141241 - Jamie Ullerich}
\fancyhead[L]{Scientific Visualisation \\ Sommersemester 2019 }
\renewcommand{\headrulewidth}{0.5pt} 				

\title{Exercise 2}

\begin{document}
	
	\maketitle 
	\thispagestyle{fancy}
	
	\section*{Exercise 2.1 - Visualization Pipeline}
	\begin{itemize}
		\item[Data acquisition] Gather GeoData, Traffic data from server
		\item[Filtering] Extract streets, streetsize/length, traffic per street, points of interest, from the data 
		\item[Mapping] Map these extracted data values to visual variables: streets to edges, Points of interest (U-Bahn etc.) to glyphs, traffic per street to color (green for little, orange for large amount of traffic)
		\item[Rendering] Bring these visual variables to the screen in a view
		\item[Interaction] The user can interact with Filtering (choose usuall traffic at specific time), Mapping (Show glyphs for U-bahn), and Rendering (Zoom, Pan...): 
		
	\end{itemize}
	
	\section*{Exercise 2.2 - Data Representation}
	
	\section*{Exercise 2.3 - Data Properties}
\end{document}
